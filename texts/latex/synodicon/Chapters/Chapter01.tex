%************************************************
\chapter{Synodicon}\label{ch:synodicon}
%************************************************
\emph{Bless, Father.}

We have received from the Church of God, that upon this day we owe yearly thanksgiving to God along with an exposition of the dogmas of piety and the overturning of the impieties of evil.

Following therefore the sayings of the prophets, honouring the exhortations of the apostles, and being instructed by the histories of the Gospels, we celebrate this day of consecration. For Esaias says: ``Be consecrated to God, ye islands,'' intimating the churches from the nations. The churches being not simply the edifices and the embellishments of the temples, but rather the congregation of the pious therein, and those who there serve the Divinity with hymns and doxologies. The Apostle advises the same thing, exhorting us, ``to walk in newness of life'' and that the ``new creation in Christ'' be renewed. So too, the oracles of the Lord prophesied our consecration. ``The consecration,'' they say, ``was in Jerusalem, and it was winter''; that is, either a spiritual winter because of the storms of bloody murder and tumult which the nation of the Jews raised against our common Saviour, or that winter which troubles the bodily senses by making the air colder. For indeed, there came upon us a winter, not an ordinary one, but one of truly great evil, brimming over with harshness; but there blossomed forth the first season, the spring of God's grace, in which we have come together to give thanks for the harvest of good things, or as we would say from the psalms, ``Summer and spring hast Thou fashioned, be mindful of this Thy creation.'' For verily, those enemies who reproached the Lord and utterly dishonoured His holy worship in the holy icons were both arrogant and high-minded in impieties, and were cast down by the God of marvels, and he leveled to the ground their insolent apostasy. Nor did He overlook the voice of those crying to Him: ``Remember, O Lord, the reproach of Thy servant which I have endured in my bosom from many nations; wherewith Thine enemies have reproached, O Lord, wherewith they have reproached the recompense of Thy Christ.'' The recompense of Christ is those who have been purchased by His death and who have believed in Him, both by the preaching of the word and by the representation in icons, whereby the redeemed know the great work of His \OE conomy\footnote{Literally, ``ruling of a house,'' thence: prudent handling, managing, thrift, good management, organization, dispensation or interposition. The word in the sacred writings refers to all of Christ's interventions for our salvation, but especially the saving dispensation of the incarnation of the Word of God, His humility, the Cross and the Resurrection.}, both the Cross and all His sufferings and miracles both before the Cross and after it; from which the imitation of His sufferings passes over unto the apostles and thence to the martyrs, and descending from them to the confessors and ascetics.

\graffito{``Without'' and ``below'' in the notes that follow are directions for the reading of the Synodicon publicly from an exedra, with the participation of organized choir, or the groups which gave the ``acclamations,'' so much a part of Byzantine public life.} This reproach wherewith the enemies of the Lord reproached, wherewith they reproached the recompense of His Christ, was remembered by God, Who was besought by His own compassion, and Who yielded to the prayers of His Mother, and moreover His apostles and all His saints who, with Him, were rendered of no account by the insolent defamation of the holy icons, so that even as the saints suffered in the flesh, so might they, as it were, suffer with Him the insults directed against the holy icons. God then wrought later that which had been counseled today, and He subsequently brought about that which He had previously performed; previously, because after many years during which the holy icons were spurned and dishonoured, He re-established true piety. But now, for a second time, after a short thirty years of harassment, He has delivered us unworthy ones from adversity, redeeming us from those who afflicted us, and establishing the free proclamation of piety, the steadfastness of the worship of icons, and this Festival which brings all of us salvation. For in the icons we see the sufferings of our Master for us---the Cross, the grave, Hades slain and pillaged---the contests of the martyrs, the crowns, that very salvation which our First Prize-giver and Contest-master and Crown-bearer wrought in the midst of the earth. This festival we celebrate today; we rejoice together and are glad with prayers and supplicatory processions, and we cry out with psalms and hymns:

What God is as great as our God? \marginpar{Without} Thou art our God, who alone worketh wonders. \numtimes{3}

For Thou didst put to scorn \marginpar{Without} those who slighted Thy Glory, and didst show forth as cowards and fugitives those who were audacious and impudent against the icons.

But thanksgiving unto God and the Master's trophy of victory against the adversaries is proper here; as for the contests and struggles against the iconoclasts, another discourse written more fully will declare them. Therefore, as a kind of rest after the desert sojourn, on the journey to reach the noetic Jerusalem, and not only in imitation of Moses, but also in obedience to the Divine Command, we considered it right as well as obligatory to inscribe on the hears of our brethren, as on a pillar constructed of large fitted stones smoothed for the reception of inscriptions, both the blessing which are due to those who keep the law, and also the curses under which transgressors put themselves. Wherefore we say thus:

To them who confess \marginpar{Without} with word, mouth, heart, and mind, and with both writing and icons the incarnate advent of God the Word,

\begin{center}
\textsc{Eternal Memory} \numtimes{3}
\end{center}

To them who acknowledge in Christ one Hypostasis, with different essences, and attribute to the one Hypostasis both the created and uncreated, the visible and invisible, the passible and impassible, the circumscribable and uncircumscribable; and then who apply on the one hand, to the Divine essence uncreatedness and the like, and, on the other hand, acknowledge with word and icons that the human nature has the other attributes accompanying circumscription,

\begin{center}
\textsc{Eternal Memory} \numtimes{3}
\end{center}

To them who believe and preach, that is proclaim, doctrines by means of writings and deeds by means of forms, and link them in a single proclamation, whereby the truth is affirmed in word and icons,

\begin{center}
\textsc{Eternal Memory} \numtimes{3}
\end{center}

To them who with words sanctify their lips, and their hearers by means of those words, and who both know and preach that the eyes of the beholders are similarly sanctified, by the venerable icons, and that through them, the mind is lifted to God-knowledge, as well as by the divine temples also, the sacred vessels, and the other precious ornaments,

\begin{center}
\textsc{Eternal Memory} \numtimes{3}
\end{center}

To them who understand that the rod and the tablets, the ark and the lampstand, and the table and the censer, from aforetime depicted and prefigured the All-Holy Virgin Mary, the Theotokos, and that these things prefigured her and not that she became these things; for she was born a maiden and remained a virgin after giving birth to God, and that for this reason she is represented as a maiden in the icons rather than obscurely depicted by types,

\begin{center}
\textsc{Eternal Memory} \numtimes{3}
\end{center}

To them who know and accept and believe concerning those things which the choir of the prophets saw, and narrated, that the Divinity Himself formed and imprinted these prophetic visions, and to those who hold fast both the written and unwritten tradition which extends from the apostles to the fathers, and who for this cause depict and honour holy things in icons,

\begin{center}
\textsc{Eternal Memory} \numtimes{3}
\end{center}

To them who understand Moses saying, ``Take heed to yourselves, that in the day when the Lord God spoke in Horeb on the mountain, ye heard the sound of words, but ye saw no likeness'' and who know to answer correctly that if we saw anything, truly did we see it, as the son of thunder has taught us, ``that which was from the beginning, which we have heard, which we have seen with our eyes, and which our hands have touched, concerning the Word of life, to these things do we bear witness''; and again as the other disciples of the Word say, ``that we both ate with Him and drank with Him, not only before the Passion, but even after the Passion and Resurrection''; to those therefore, who have been strengthened by God to distinguish between the commandment in the Law and the teaching which came with Grace, and between that which was invisible in the former, but both visible and tangible in the latter, and who for this cause depict and worship in icons these visible and tangible realities,

\begin{center}
\textsc{Eternal Memory} \numtimes{3}
\end{center}

As the prophets have seen, as the apostles have taught, as the Church has received, as the teachers have set forth in dogmas, as the whole world has understood, as Grace has shone forth, as the truth was demonstrated, as falsehood was banished, as wisdom was emboldened, as Christ has awarded; thus do we believe, thus we speak, thus we preach Christ our true God and His saints, honouring them in words, in writings, in thoughts, in sacrifices, in temples, and in icons, worshipping and respecting the One as God and Master, and honouring the others, and apportioning relative worship to them, because of our common Master for they are His genuine servants,

This is the Faith of the apostles, \marginpar{Without} this is the Faith of the fathers, this is the Faith of the Orthodox, this Faith hath established the whole world.

We now take occasion to acclaim fraternally and with filial affection the preachers of piety unto the glory and honour of godliness, for which they struggled, and we say,

To Germanus, Tarasius, Nicephorus \marginpar{Below} and Methodius who are truly high priests of God and champions and teachers of Orthodoxy,

\begin{center}
\textsc{Eternal Memory} \numtimes{3}
\end{center}

To Ignatius, Photius, Stephen, Anthony, and Nicholas the most holy and Orthodox patriarchs,

\begin{center}
\textsc{Eternal Memory} \numtimes{3}
\end{center}

All that was written or spoken against the holy Patriarchs Germanus, Tarasius, Nicephorus, and Methodius, Ignatius, Photius, Nicephorus, Anthony and Nicholas,

\begin{center}
\textsc{Anathema} \numtimes{3}
\end{center}
